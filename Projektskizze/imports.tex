

%%%%%%%%%%%%%%%%%%%%%%%%%%%%%%%%%%%%%%%%%%%%%%%%%%%%%%%%%%%
% Obere Titelmakros. Editieren Sie diese Datei nur, wenn
% Sie sich ABSOLUT sicher sind, was Sie da tun!!!
% (Z.B. zum Abaendern der BA-Vorlage in eine MA-Vorlage)
% Uni Duesseldorf
% Lehrstuhl fuer Datenbanken und Informationssysteme
% Version 2.2 - 2.3.2010
%%%%%%%%%%%%%%%%%%%%%%%%%%%%%%%%%%%%%%%%%%%%%%%%%%%%%%%%%%%
\newcommand{\AN}{twoside}
\newcommand{\AUS}{}


%\newcommand{\englisch}{}
%\newcommand{\deutsch}{\usepackage[german]{babel}}

%% Die folgenden auskommentierten Optionen dienen der automatischen
%% Erkennung des Latex-Kompilers und dem Setzen der davon abhängigen
%% Einstellungen. Bei Problem z.B. mit dem Einbinden von verschiedenen
%% Grafiktypen bei Verwendung von PdfLatex oder Latex, einfach die
%% verschiedenen \usepackage(s) ausprobieren. (Mit diesen Einstellungen
%% funktionierte diese Vorlage bei der Verwenundg von latex.exe als
%% Kompiler bei den meisten Studierenden.)

%\newif\ifpdf \ifx\pdfoutput\undefined
%\pdffalse % we are not running pdflatex
%\else
%\pdfoutput=1 % we are running pdflatex
%\pdfcompresslevel=9 % compression level for text and image;
%\pdftrue \fi

\documentclass[11pt,a4paper, \zweiseitig]{article}
\usepackage{ifthen}


%\usepackage[iso]{umlaute}
\usepackage[utf8]{inputenc}
\usepackage{libertine} % palatino Schriftart
%\usepackage{makeidx} % um ein Index zu erstellen
\usepackage[nottoc]{tocbibind}
\usepackage[T1]{fontenc} %fuer richtige Trennung bei Umlauten
\usepackage{fancybox} % fuer die Rahmen
\usepackage{shortvrb}
\usepackage{url}
\usepackage{xcolor}
\usepackage[colorlinks,citecolor=blue,linkcolor=black]{hyperref} %anklickbares Inhaltsverzeichnis

\ifthenelse{\boolean{\biber}}{
  % only needed for biber
  \usepackage[style=authoryear,natbib=true,backend=biber,mincitenames=1,maxcitenames=2,maxbibnames=99,uniquelist=false,dashed=false]{biblatex}

  % https://tex.stackexchange.com/a/334703/8850
  \AtEveryBibitem{%
    \clearfield{issn}
    \clearfield{isbn}
    \clearfield{doi}
    \clearfield{location}
    \clearlist{location}
    \clearlist{address}

    \ifentrytype{online}{}{% Remove url except for @online
      \clearfield{url}
    }
  }
}
{}%no else

% Falls es bei \citet ein Komma zwischen Name und Jahr gibt:
% https://tex.stackexchange.com/questions/312539/unwanted-comma-between-author-and-year-using-citet-command
% (thx @ Markus Brenneis)
%\DeclareDelimFormat[cbx@textcite]{nameyeardelim}{\addspace}



\ifthenelse{\equal{\sprache}{deutsch}}{
  \usepackage[ngerman]{babel}
  % Bibtex u.a -> et al.
  \ifthenelse{\boolean{\biber}}{
    \DefineBibliographyStrings{ngerman}{
      andothers = {{et\,al\adddot}},
    }
    \newcommand{\references}{Literatur}
  }
  {} % do nothing when not using biber
  \usepackage[autostyle, german=quotes]{csquotes} % Deutsche Anführungszeichen im Literaturverzeichnis (thx @ Markus Brenneis)

}{ \newcommand{\references}{References}}

\usepackage{a4wide} % ganze A4 Weite verwenden



%\ifpdf
%\usepackage[pdftex,xdvi]{graphicx}
%\usepackage{thumbpdf} %thumbs fuer Pdf
%\usepackage[pdfstartview=FitV]{hyperref} %anklickbares Inhaltsverzeichnis
%\else
%\usepackage[dvips,xdvi]{graphicx}
\usepackage{graphicx}

%\fi

\newcommand{\redt}[1] {
  \textcolor{red}{#1}}

\newcommand{\oranget}[1] {
  \textcolor{orange}{#1}}

\newcommand{\purplet}[1] {
  \textcolor{purple}{#1}}

%%%%%%%%%%%%%%%%%%%%%%% Massangaben fuer die Arbeit %%%%%%%%%%%%%%%
\setlength{\textwidth}{15cm}

\setlength{\oddsidemargin}{35mm}
\setlength{\evensidemargin}{25mm}

\addtolength{\oddsidemargin}{-1in}
\addtolength{\evensidemargin}{-1in}

\ifthenelse{\boolean{\biber}}{\addbibresource{references.bib}}{}

%\makeindex