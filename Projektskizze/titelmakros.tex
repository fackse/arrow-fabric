%\setcounter{secnumdepth}{4} %Nummerieren bis in die 4. Ebene
%\setcounter{tocdepth}{4} %Inhaltsverzeichnis bis zur 4. Ebene

\pagestyle{headings}

\sloppy % LaTeX ist dann nicht so streng mit der Silbentrennung
%~ \MakeShortVerb{\§}

\parindent0mm
\parskip0.5em


{
\textwidth170mm
\oddsidemargin30mm
\evensidemargin30mm
\addtolength{\oddsidemargin}{-1in}
\addtolength{\evensidemargin}{-1in}

\parskip0pt plus2pt

% Die Raender muessen eventuell fuer jeden Drucker individuell eingestellt
% werden. Dazu sind die Werte fuer die Abstaende `\oben' und `\links' zu
% aendern, die von mir auf jeweils 0mm eingestellt wurden.

%\newlength{\links} \setlength{\links}{10mm}  % hier abzuaendern
%\addtolength{\oddsidemargin}{\links}
%\addtolength{\evensidemargin}{\links}

\begin{titlepage}
\vspace*{-1.5cm}
\raisebox{17mm}{
    \begin{minipage}[t]{70mm}
        \begin{center}
            %\selectlanguage{german}
            {\Large INSTITUT FÜR INFORMATIK\\}
            {\normalsize
                Betriebssysteme\\
            }
            \vspace{3mm}
            {\small Universitätsstr. 1 \hspace{5ex} D--40225 Düsseldorf\\}
        \end{center}
    \end{minipage}
}
\hfill
\raisebox{7mm}{
    \includegraphics[width=130pt]{bilder/HHU_Logo}}
\vspace{14em}

% Titel
\begin{center}
    \baselineskip=55pt
    \textbf{\huge \titel}
    \baselineskip=0 pt
\end{center}

%\vspace{7em}

\vfill

% Autor
\begin{center}
    \textbf{\Large
        \bearbeiter
    }
\end{center}

\vspace{35mm}

% Prüfungsordnungs-Angaben
\begin{center}
%\selectlanguage{german}


\vspace{2em}
\ifthenelse{\equal{\sprache}{deutsch}}{
    \begin{tabular}[t]{ll}
    Datum:& {\today} \\
    
\end{tabular}
\end{center}

\end{titlepage}

}

%%%%%%%%%%%%%%%%%%%%%%%%%%%%%%%%%%%%%%%%%%%%%%%%%%%%%%%%%%%%%%%%%%%%%
\clearpage
\begin{titlepage}
    ~                % eine leere Seite hinter dem Deckblatt
\end{titlepage}
%%%%%%%%%%%%%%%%%%%%%%%%%%%%%%%%%%%%%%%%%%%%%%%%%%%%%%%%%%%%%%%%%%%%%

%%%%%%%%%%%%%%%%%%%%%%%%%%%%%%%%%%%%%%%%%%%%%%%%%%%%%%%%%%%%%%%%%%%%%
% Leerseite bei zweiseitigem Druck
%%%%%%%%%%%%%%%%%%%%%%%%%%%%%%%%%%%%%%%%%%%%%%%%%%%%%%%%%%%%%%%%%%%%%

\ifthenelse{\equal{\zweiseitig}{twoside}}{\clearpage\begin{titlepage}
        ~\end{titlepage}}{}

%%%%%%%%%%%%%%%%%%%%%%%%%%%%%%%%%%%%%%%%%%%%%%%%%%%%%%%%%%%%%%%%%%%%%


%%%%%%%%%%%%%%%%%%%%%%%%%%%%%%%%%%%%%%%%%%%%%%%%%%%%%%%%%%%%%%%%%%%%%
% Leerseite bei zweiseitigem Druck
%%%%%%%%%%%%%%%%%%%%%%%%%%%%%%%%%%%%%%%%%%%%%%%%%%%%%%%%%%%%%%%%%%%%%
\ifthenelse{\equal{\zweiseitig}{twoside}}
{\clearpage\begin{titlepage}~\end{titlepage}}{}
%%%%%%%%%%%%%%%%%%%%%%%%%%%%%%%%%%%%%%%%%%%%%%%%%%%%%%%%%%%%%%%%%%%%%
\clearpage \setcounter{page}{1}
\pagenumbering{roman}
\setcounter{tocdepth}{2}
\tableofcontents

%\enlargethispage{\baselineskip}
\clearpage
%%%%%%%%%%%%%%%%%%%%%%%%%%%%%%%%%%%%%%%%%%%%%%%%%%%%%%%%%%%%%%%%%%%%%
% Leere Seite, falls Inhaltsverzeichnis mit ungerader Seitenzahl und
% doppelseitiger Druck
%%%%%%%%%%%%%%%%%%%%%%%%%%%%%%%%%%%%%%%%%%%%%%%%%%%%%%%%%%%%%%%%%%%%%
\ifthenelse{ \( \equal{\zweiseitig}{twoside} \and \not \isodd{\value{page}} \)}
{\pagebreak \thispagestyle{empty} \cleardoublepage}{\clearpage}


% Kapitel soll bei doppelseitigem Druck immer auf der rechten (ungeraden) Seite anfangen (thx @ Philipp Grawe)
% https://tex.stackexchange.com/a/223387
\ifthenelse{\boolean{\sectionforcestartright}}
{\let\oldsection\section % Store \section in \oldsection
    \renewcommand{\section}{\cleardoublepage\oldsection}}
{}